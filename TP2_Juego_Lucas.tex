% % Apunte de modulos basicos
%
\documentclass[a4paper,10pt]{article}
\usepackage[paper=a4paper, hmargin=1.5cm, bottom=1.5cm, top=3.5cm]{geometry}
\usepackage[latin1]{inputenc}
\usepackage[T1]{fontenc}
\usepackage[spanish]{babel}
\usepackage{fancyhdr}
\usepackage{lastpage}
\usepackage{xspace}
\usepackage{xargs}
\usepackage{ifthen}
\usepackage{aed2-tad,aed2-symb,aed2-itef}
\usepackage{algorithmicx, algpseudocode, algorithm}
\usepackage[colorlinks=true, linkcolor=blue]{hyperref}


\hypersetup{%
 % Para que el PDF se abra a p�gina completa.
 pdfstartview= {FitH \hypercalcbp{\paperheight-\topmargin-1in-\headheight}},
 pdfauthor={C�tedra de Algoritmos y Estructuras de Datos II - DC - UBA},
 pdfkeywords={M�dulos b�sicos},
 pdftitle={M�dulos b�sicos de dise�o},
 pdfsubject={M�dulos b�sicos de dise�o}
}

%%%%%%%%%%%%%%%%%%%%%%%%%%%%%%%%%%%%%%%%%%%%%%%%%
% PARAMETROS A SER MODIFICADOS
%%%%%%%%%%%%%%%%%%%%%%%%%%%%%%%%%%%%%%%%%%%%%%%%%

%cuatrimestre de acuerdo a la opcion
\newcommand{\Cuatrimestre}{$2^\mathrm{do}$ cuatrimestre de 2016 (compilado?)}
%\newcommand{\Cuatrimestre}{$1^\mathrm{er}$ cuatrimestre de 2012 (compilado 08/05/2012)}
%\renewcommand{\Cuatrimestre}{$2^\mathrm{do}$ cuatrimestre de 2010 (compilado 13/05/2011)}

%%%%%%%%%%%%%%%%%%%%%%%%%%%%%%%%%%%%%%%%%%%%%%%%%
% OTRAS OPCIONES QUE NO HAY QUE MODIFICAR
%%%%%%%%%%%%%%%%%%%%%%%%%%%%%%%%%%%%%%%%%%%%%%%%%

%opening
\title{Apunte de M�dulos B�sicos (v.\ 0.3$\alpha$)}
\author{Algoritmos y Estructuras de Datos II, DC, UBA.}
\date{\Cuatrimestre}

% Acomodo fancyhdr.
\pagestyle{fancy}
\thispagestyle{fancy}
\lhead{Algoritmos y Estructuras de Datos II}
\rhead{\Cuatrimestre}
\cfoot{\thepage /\pageref{LastPage}}
\renewcommand{\footrulewidth}{0.4pt}
\setlength{\headheight}{13pt}

%%%%%%%%%%%%%%%%%%%%%%%%%%%%%%%%%%%%%%%%%%%%%%%%%%%%%%%%%%%%
% COMANDOS QUE ALGUN DIA PUEDAN FORMAR UN PAQUETE.
%%%%%%%%%%%%%%%%%%%%%%%%%%%%%%%%%%%%%%%%%%%%%%%%%%%%%%%%%%%%
\newcommand{\moduloNombre}[1]{\textbf{#1}}

\let\NombreFuncion=\textsc
\let\TipoVariable=\texttt
\let\ModificadorArgumento=\textbf
\newcommand{\res}{$res$\xspace}
\newcommand{\tab}{\hspace*{7mm}}

\newcommandx{\TipoFuncion}[3]{%
  \NombreFuncion{#1}(#2) \ifx#3\empty\else $\to$ \res\,: \TipoVariable{#3}\fi%
}
\newcommand{\In}[2]{\ModificadorArgumento{in} \ensuremath{#1}\,: \TipoVariable{#2}\xspace}
\newcommand{\Out}[2]{\ModificadorArgumento{out} \ensuremath{#1}\,: \TipoVariable{#2}\xspace}
\newcommand{\Inout}[2]{\ModificadorArgumento{in/out} \ensuremath{#1}\,: \TipoVariable{#2}\xspace}
\newcommand{\Aplicar}[2]{\NombreFuncion{#1}(#2)}

\newlength{\IntFuncionLengthA}
\newlength{\IntFuncionLengthB}
\newlength{\IntFuncionLengthC}
%InterfazFuncion(nombre, argumentos, valor retorno, precondicion, postcondicion, complejidad, descripcion, aliasing)
\newcommandx{\InterfazFuncion}[9][4=true,6,7,8,9]{%
  \hangindent=\parindent
  \TipoFuncion{#1}{#2}{#3}\\%
  \textbf{Pre} $\equiv$ \{#4\}\\%
  \textbf{Post} $\equiv$ \{#5\}%
  \ifx#6\empty\else\\\textbf{Complejidad:} #6\fi%
  \ifx#7\empty\else\\\textbf{Descripci�n:} #7\fi%
  \ifx#8\empty\else\\\textbf{Aliasing:} #8\fi%
  \ifx#9\empty\else\\\textbf{Requiere:} #9\fi%
}

\newenvironment{Interfaz}{%
  \parskip=2ex%
  \noindent\textbf{\Large Interfaz}%
  \par%
}{}

\newenvironment{Representacion}{%
  \vspace*{2ex}%
  \noindent\textbf{\Large Representaci�n}%
  \vspace*{2ex}%
}{}

\newenvironment{Algoritmos}{%
  \vspace*{2ex}%
  \noindent\textbf{\Large Algoritmos}%
  \vspace*{2ex}%
}{}


\newcommand{\Titulo}[1]{
  \vspace*{1ex}\par\noindent\textbf{\large #1}\par
}

\newenvironmentx{Estructura}[2][2={estr}]{%
  \par\vspace*{2ex}%
  \TipoVariable{#1} \textbf{se representa con} \TipoVariable{#2}%
  \par\vspace*{1ex}%
}{%
  \par\vspace*{2ex}%
}%

\newboolean{EstructuraHayItems}
\newlength{\lenTupla}
\newenvironmentx{Tupla}[1][1={estr}]{%
    \settowidth{\lenTupla}{\hspace*{3mm}donde \TipoVariable{#1} es \TipoVariable{tupla}$($}%
    \addtolength{\lenTupla}{\parindent}%
    \hspace*{3mm}donde \TipoVariable{#1} es \TipoVariable{tupla}$($%
    \begin{minipage}[t]{\linewidth-\lenTupla}%
    \setboolean{EstructuraHayItems}{false}%
}{%
    $)$%
    \end{minipage}
}

\newcommandx{\tupItem}[3][1={\ }]{%
    %\hspace*{3mm}%
    \ifthenelse{\boolean{EstructuraHayItems}}{%
        ,#1%
    }{}%
    \emph{#2}: \TipoVariable{#3}%
    \setboolean{EstructuraHayItems}{true}%
}

\newcommandx{\RepFc}[3][1={estr},2={e}]{%
  \tadOperacion{Rep}{#1}{bool}{}%
  \tadAxioma{Rep($#2$)}{#3}%
}%

\newcommandx{\Rep}[3][1={estr},2={e}]{%
  \tadOperacion{Rep}{#1}{bool}{}%
  \tadAxioma{Rep($#2$)}{true \ssi #3}%
}%

\newcommandx{\Abs}[5][1={estr},3={e}]{%
  \tadOperacion{Abs}{#1/#3}{#2}{Rep($#3$)}%
  \settominwidth{\hangindent}{Abs($#3$) \igobs #4: #2 $\mid$ }%
  \addtolength{\hangindent}{\parindent}%
  Abs($#3$) \igobs #4: #2 $\mid$ #5%
}%

\newcommandx{\AbsFc}[4][1={estr},3={e}]{%
  \tadOperacion{Abs}{#1/#3}{#2}{Rep($#3$)}%
  \tadAxioma{Abs($#3$)}{#4}%
}%


\newcommand{\DRef}{\ensuremath{\rightarrow}}

\begin{document}

%pagina de titulo
\thispagestyle{empty}
\maketitle
\tableofcontents

\newpage


\section{M�dulo Juego}

Descripci�n del m�dulo Juego.

\begin{Interfaz}
  
  \Titulo{Operaciones b�sicas de juego}
  
  \InterfazFuncion{Pok�mons}{\In{e}{jugador}, \In{j}{juego}}{itConj(<pok�mon, nat>)}
  [$e \in jugadore(j)$]
  {res = CrearIt(clasificarPokemons(pok�mons(e, j)))}
  [$\Theta(1)$]
  [Retorna un iterador al conjunto de tuplas <pok�mons, cantidad> del jugador e.]
  [El iterador se invalida si y s�lo si se elimina el elemento siguiente del iterador sin utilizar la funci�n \NombreFuncion{EliminarSiguiente}.]

  \InterfazFuncion{Expulsados}{\In{j}{juego}}{conj(jugador)}
  {alias($res$, expulsados(j))}
  [$\Theta(1)$]
  [Retorna el conjunto de jugador expulsados por referencia.]
  [ ?? ]
  
  \InterfazFuncion{PosConPok�mons}{\In{j}{juego}}{conj(coor)}
  {alias($res$, posConPok�mons(j))}
  [$\Theta(1)$]
  [Retorna el conjunto de coordenadas donde hay pok�mons por referencia.]
  [ ?? ]
  
  \InterfazFuncion{Pok�monEnPos}{\In{c}{coor}, \In{j}{juego}}{pok�mon}
  [$c \in$ posConPok�mons(j)]
  {$res$ $=$ pok�monEnPos(c, j))}
  [$\Theta(??)$]
  [Retorna por copia el pok�mon que se encuentra en la coordenada c.]

  \InterfazFuncion{CantMovimientosParaCaptura}{\In{c}{coor}, \In{j}{juego}}{nat}
  [$c$ $\in$ posConPok�mons(j)]
  {$res$ $=$ cantMovimientosParaCaptura(c, j))}
  [$\Theta(??)$]
  [Retorna por copia la cantidad de movimientos para capturar al pok�mon que se encuentra en la coordenada c.]

  \InterfazFuncion{CantMismaEspecie}{\In{p}{pok�mon}, \In{cp}{multiconj(pok�mon)}}{nat}
  {$res$ $=$ cantMismaEspecie(p, cp)}
  [$\Theta(??))$]
  [Retorna la cantidad de pok�mons de la especie p que hay en el conjunto cp.]

  
  \Titulo{Especificaci�n de las operaciones auxiliares utilizadas en la interfaz}
  
  \begin{tad}{Juego Extendido}
    \parskip=0pt
    \tadExtiende{\tadNombre{Juego}}
    
    \tadTitulo{otras operaciones (no exportadas)}
    
    \tadAlinearFunciones{clasificarPokemons}{multiconj(pok�mon), pok�mon}
    \tadOperacion{clasificarPokemons}{multiconj(pok�mon)}{conj(<pok�mon, nat>)}{}
    \tadOperacion{eliminar}{multiconj(pok�mon), pok�mon}{multiconj(pok�mon)}{}
    
    \tadAxiomas
    \tadAlinearAxiomas{clasificarPokemons(m)}
    \tadAxioma{clasificarPokemons(m)}{\IF $\emptyset$(m) THEN $\emptyset$ ELSE Ag(<dameUno(m), $\#$dameUno(m)>, \\clasificarPok�mons(eliminar(m, dameUno(m)))) FI}
    \tadAxioma{eliminar(m, p)}{\IF $\emptyset$(m) THEN $\emptyset$ ELSE {\IF dameUno(m) = p THEN eliminar(sinUno(m), p) ELSE Ag(dameUno(m), eliminar(sinUno(m), p)) FI} FI}
  \end{tad}
  
  
\end{Interfaz}

\end{document}
