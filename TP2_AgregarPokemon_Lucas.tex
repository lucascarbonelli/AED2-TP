% % Apunte de modulos basicos
%
\documentclass[a4paper,10pt]{article}
\usepackage[paper=a4paper, hmargin=1.5cm, bottom=1.5cm, top=3.5cm]{geometry}
\usepackage[latin1]{inputenc}
\usepackage[T1]{fontenc}
\usepackage[spanish]{babel}
\usepackage{fancyhdr}
\usepackage{lastpage}
\usepackage{xspace}
\usepackage{xargs}
\usepackage{ifthen}
\usepackage{aed2-tad,aed2-symb,aed2-itef}
\usepackage{algorithmicx, algpseudocode, algorithm}
\usepackage[colorlinks=true, linkcolor=blue]{hyperref}


\hypersetup{%
 % Para que el PDF se abra a p�gina completa.
 pdfstartview= {FitH \hypercalcbp{\paperheight-\topmargin-1in-\headheight}},
 pdfauthor={C�tedra de Algoritmos y Estructuras de Datos II - DC - UBA},
 pdfkeywords={M�dulos b�sicos},
 pdftitle={M�dulos b�sicos de dise�o},
 pdfsubject={M�dulos b�sicos de dise�o}
}

%%%%%%%%%%%%%%%%%%%%%%%%%%%%%%%%%%%%%%%%%%%%%%%%%
% PARAMETROS A SER MODIFICADOS
%%%%%%%%%%%%%%%%%%%%%%%%%%%%%%%%%%%%%%%%%%%%%%%%%

%cuatrimestre de acuerdo a la opcion
\newcommand{\Cuatrimestre}{$2^\mathrm{do}$ cuatrimestre de 2016 (compilado?)}
%\newcommand{\Cuatrimestre}{$1^\mathrm{er}$ cuatrimestre de 2012 (compilado 08/05/2012)}
%\renewcommand{\Cuatrimestre}{$2^\mathrm{do}$ cuatrimestre de 2010 (compilado 13/05/2011)}

%%%%%%%%%%%%%%%%%%%%%%%%%%%%%%%%%%%%%%%%%%%%%%%%%
% OTRAS OPCIONES QUE NO HAY QUE MODIFICAR
%%%%%%%%%%%%%%%%%%%%%%%%%%%%%%%%%%%%%%%%%%%%%%%%%

%opening
\title{Apunte de M�dulos B�sicos (v.\ 0.3$\alpha$)}
\author{Algoritmos y Estructuras de Datos II, DC, UBA.}
\date{\Cuatrimestre}

% Acomodo fancyhdr.
\pagestyle{fancy}
\thispagestyle{fancy}
\lhead{Algoritmos y Estructuras de Datos II}
\rhead{\Cuatrimestre}
\cfoot{\thepage /\pageref{LastPage}}
\renewcommand{\footrulewidth}{0.4pt}
\setlength{\headheight}{13pt}

%%%%%%%%%%%%%%%%%%%%%%%%%%%%%%%%%%%%%%%%%%%%%%%%%%%%%%%%%%%%
% COMANDOS QUE ALGUN DIA PUEDAN FORMAR UN PAQUETE.
%%%%%%%%%%%%%%%%%%%%%%%%%%%%%%%%%%%%%%%%%%%%%%%%%%%%%%%%%%%%
\newcommand{\moduloNombre}[1]{\textbf{#1}}

\let\NombreFuncion=\textsc
\let\TipoVariable=\texttt
\let\ModificadorArgumento=\textbf
\newcommand{\res}{$res$\xspace}
\newcommand{\tab}{\hspace*{7mm}}

\newcommandx{\TipoFuncion}[3]{%
  \NombreFuncion{#1}(#2) \ifx#3\empty\else $\to$ \res\,: \TipoVariable{#3}\fi%
}
\newcommand{\In}[2]{\ModificadorArgumento{in} \ensuremath{#1}\,: \TipoVariable{#2}\xspace}
\newcommand{\Out}[2]{\ModificadorArgumento{out} \ensuremath{#1}\,: \TipoVariable{#2}\xspace}
\newcommand{\Inout}[2]{\ModificadorArgumento{in/out} \ensuremath{#1}\,: \TipoVariable{#2}\xspace}
\newcommand{\Aplicar}[2]{\NombreFuncion{#1}(#2)}

\newlength{\IntFuncionLengthA}
\newlength{\IntFuncionLengthB}
\newlength{\IntFuncionLengthC}
%InterfazFuncion(nombre, argumentos, valor retorno, precondicion, postcondicion, complejidad, descripcion, aliasing)
\newcommandx{\InterfazFuncion}[9][4=true,6,7,8,9]{%
  \hangindent=\parindent
  \TipoFuncion{#1}{#2}{#3}\\%
  \textbf{Pre} $\equiv$ \{#4\}\\%
  \textbf{Post} $\equiv$ \{#5\}%
  \ifx#6\empty\else\\\textbf{Complejidad:} #6\fi%
  \ifx#7\empty\else\\\textbf{Descripci�n:} #7\fi%
  \ifx#8\empty\else\\\textbf{Aliasing:} #8\fi%
  \ifx#9\empty\else\\\textbf{Requiere:} #9\fi%
}

\newenvironment{Interfaz}{%
  \parskip=2ex%
  \noindent\textbf{\Large Interfaz}%
  \par%
}{}

\newenvironment{Representacion}{%
  \vspace*{2ex}%
  \noindent\textbf{\Large Representaci�n}%
  \vspace*{2ex}%
}{}

\newenvironment{Algoritmos}{%
  \vspace*{2ex}%
  \noindent\textbf{\Large Algoritmos}%
  \vspace*{2ex}%
}{}


\newcommand{\Titulo}[1]{
  \vspace*{1ex}\par\noindent\textbf{\large #1}\par
}

\newenvironmentx{Estructura}[2][2={estr}]{%
  \par\vspace*{2ex}%
  \TipoVariable{#1} \textbf{se representa con} \TipoVariable{#2}%
  \par\vspace*{1ex}%
}{%
  \par\vspace*{2ex}%
}%

\newboolean{EstructuraHayItems}
\newlength{\lenTupla}
\newenvironmentx{Tupla}[1][1={estr}]{%
    \settowidth{\lenTupla}{\hspace*{3mm}donde \TipoVariable{#1} es \TipoVariable{tupla}$($}%
    \addtolength{\lenTupla}{\parindent}%
    \hspace*{3mm}donde \TipoVariable{#1} es \TipoVariable{tupla}$($%
    \begin{minipage}[t]{\linewidth-\lenTupla}%
    \setboolean{EstructuraHayItems}{false}%
}{%
    $)$%
    \end{minipage}
}

\newcommandx{\tupItem}[3][1={\ }]{%
    %\hspace*{3mm}%
    \ifthenelse{\boolean{EstructuraHayItems}}{%
        ,#1%
    }{}%
    \emph{#2}: \TipoVariable{#3}%
    \setboolean{EstructuraHayItems}{true}%
}

\newcommandx{\RepFc}[3][1={estr},2={e}]{%
  \tadOperacion{Rep}{#1}{bool}{}%
  \tadAxioma{Rep($#2$)}{#3}%
}%

\newcommandx{\Rep}[3][1={estr},2={e}]{%
  \tadOperacion{Rep}{#1}{bool}{}%
  \tadAxioma{Rep($#2$)}{true \ssi #3}%
}%

\newcommandx{\Abs}[5][1={estr},3={e}]{%
  \tadOperacion{Abs}{#1/#3}{#2}{Rep($#3$)}%
  \settominwidth{\hangindent}{Abs($#3$) \igobs #4: #2 $\mid$ }%
  \addtolength{\hangindent}{\parindent}%
  Abs($#3$) \igobs #4: #2 $\mid$ #5%
}%

\newcommandx{\AbsFc}[4][1={estr},3={e}]{%
  \tadOperacion{Abs}{#1/#3}{#2}{Rep($#3$)}%
  \tadAxioma{Abs($#3$)}{#4}%
}%


\newcommand{\DRef}{\ensuremath{\rightarrow}}

\begin{document}

%pagina de titulo
\thispagestyle{empty}
%\maketitle
%\tableofcontents

%\newpage

\section{Juego}

\begin{Interfaz}
  
  \InterfazFuncion{AgregarPok�mon}{\In{p}{pok�mon}, \In{c}{coor}, \Inout{j}{juego}}{}
  [$j \igobs j_0 \land puedoAgregarPok�mon(c, j)$]
  {c $\igobs$ agregarPok�mon($c_0$)}
  [$\Theta(|P| + EC * log(EC))$]
  [Agrega al juego en la coordenada $c$ un pok�mon de tipo $p$.]
  
\end{Interfaz}


%%%%%%%%%%%%%%%%%%%%%%%%%%%%%%%%%%%%%%%%%%%%%%%%%%%%%%%%%%%%%%%%

\begin{Algoritmos}

\medskip
	
 \Titulo{Algoritmos del m�dulo}
  	\medskip
  
\begin{algorithm}[H]{\textbf{iAgregarPok�mon}(\In{p}{pok�mon}, \In{c}{coor}, \Inout{j}{juego})}
    	\begin{algorithmic}[1]
	\State $itHeapmod \ itNuevoHeap \gets iCrearHeapPokemon(j, c)$ \Comment {Se crea el heap que contiene a todos los jugadores en radio 2 del pok�mon} // $\Theta(EC * log(EC))$

 	\State $itDic \ dicLinealPoke \gets iDefinirRapido(j.posPokemon, c, \langle p, 0, itNuevoHeap \rangle)$ \Comment {Se define la coordenada del pok�mon y se coloca su infoCoord} // $\Theta(1)$
 			 
	\State $j.matrizPokemons[iLongitud(c)][iLatitud(c)] \gets \langle true, dicLinealPoke \rangle$ \Comment {Se cambia $true$ la posici�n donde est� el pok�mon y se pone un iterador al diccionario de coordenadas} // $\Theta(1)$
			 
    	 \If{$iDefinido?(j.pokemonsTotales, p)$} \Comment {Se verifica que est� definido el tipo pok�mon $p$} // $\Theta(iLongitud(p))$
    	 
    	 %	\State $ $ \Comment {Se incremente la cantidad de pokemones del tipo agregado} // $\Theta(iLongitud(p))$
    	 
		\State $iObtener(j.pokemonsTotales, p).cant \gets iObtener(j.pokemonsTotales, p).cant + 1$ \Comment {Se incremente la cantidad de pokemones del tipo agregado} // $\Theta(iLongitud(p))$
				
		\State $AgrgarRapido(iObtener(j.pokemonsTotales, p).pos, dicLinealPoke)$ \Comment {Se agrega el iterador al conjunto de posiciones} // $\Theta(iLongitud(p))$
			
    	\Else
    	
	    	\State $iDefinir(j.pokemonsTotales, p, \langle 1, iVacio() \rangle)$	\Comment {Se define el nuevo tipo de pokemon} // $\Theta(iLongitud(p))$
	    	
		\State $AgrgarRapido(iObtener(j.pokemonsTotales, p).pos, dicLinealPoke)$ \Comment {Se agrega el iterador al conjunto de posiciones} // $\Theta(iLongitud(p))$

	\EndIf
			
			\medskip
			\Statex \underline{Complejidad:} $\Theta(|P| + EC * log(EC))$
			\Statex \underline{Justificaci�n:} El algoritmo cuenta con una operaci�n que tiene costo $\Theta(EC * log(EC))$, tres operaciones con costo $\Theta(|P|)$ y otra operaciones con costo $\Theta(1)$. Luego sumando las complejidades se obtiene que la complejidad del algoritmo es $\Theta(|P| + EC * log(EC))$.
    	\end{algorithmic}
\end{algorithm}


\begin{algorithm}[H]{\textbf{iCrearHeapPokemon}(\In{j}{juego}, \In{c}{coor}) $\to$ $res$ : heapMod($\alpha$)}
    	\begin{algorithmic}[1]
    	\State $heapmod \ pokeHeap \gets Vacia()$ \Comment // $\Theta(1)$
	\State $i \gets -2$ \Comment // $\Theta(1)$
	\While {$i \leq 2$} \Comment // $\Theta(1)$
	
		\State $j \gets -2$ \Comment // $\Theta(1)$
		\While {$j \leq 2$} \Comment // $\Theta(1)$
		
			\State $coor \ coorJug \gets \langle longitud(c)+i, latitud(c)+j \rangle$ \Comment // $\Theta(1)$
			
			\If {$posExistente(coorJug, j.mapa) \land DistEuclidea(c, coorJug) \leq 2$} \Comment {Vemos que la coordenada exista en el mapa y est� en rango del pok�mon} // $\Theta(1)$
			\State $itConj(Jugador) \ itJugHeap \gets CrearIt(j.matrizJugadores[i][j])$ \Comment // $\Theta(1)$
			
				\While {$HaySiguiente(itJugHeap)$} \Comment {Analizamos los jugadores que hay en esa coordenada} // $\Theta(1)$
					\State $id \gets Siguiente(itJugHeap)$ \Comment $\Theta(1)$
					\State $itHeapmod \ itPokeHeap \gets Encolar(\langle \#(jugadores[id].pokemons), id \rangle, pokeHeap)$ \Comment {Agregamos al jugador a la cola de prioridad del pok�mon y creamos un iterador al heap} // $\Theta(log(EC))$
					\State $jugadores[id].posicionCola \gets itPokeHeap$ \Comment {Colocamos el iterador info del jugador} // $\Theta(1)$
					\State $Avanzar(itJugHeap)$ \Comment // $\Theta(1)$
	
				\EndWhile
			\EndIf
			
			\State $j \gets j + 1$ \Comment // $\Theta(1)$
			
		\EndWhile
		\State $i \gets i + 1$ \Comment // $\Theta(1)$
		
	\EndWhile
	
			\medskip
			\Statex \underline{Complejidad:} $\Theta(EC * log(EC))$
			\Statex \underline{Justificaci�n:} Todas las operaciones tienen costo $\Theta(1)$, excepto $Encolar$ que tiene costo $\Theta(log(EC))$, donde $EC$ es la cantidad de elementos en la cola. El algoritmo tiene dos ciclos, pero tienen una cantidad predeterminada de iteraciones, tanto $i$ como $j$ toma valores en el intervalo [-2,2]. Luego la complejidad depende de la cantidad de elementos($jugadores$) que se coloquen en la cola. Denominamos $EC$ a la cantidad de jugadores que se encuentran en rango del pok�mon y deben ser colocados en la cola de prioridad. Ya que $Encolar$ un jugador que tiene costo $\Theta(log(EC))$ y hay $EC$ jugadores, la complejidad es $\Theta(EC * log(EC))$.
    	\end{algorithmic}
\end{algorithm}

 
\end{Algoritmos}

\end{document}
